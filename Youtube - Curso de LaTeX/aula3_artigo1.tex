\documentclass[12pt, a4paper]{article}
\usepackage[brazilian]{babel} %Traduz o documento para o português brasileiro.
\usepackage[utf8]{inputenc} %Reconhece à acentuação
\usepackage{lipsum} %Gera texto aleatório
\usepackage{ragged2e} %Mais opções de alinhamento
\newtheorem{teorema}{Teorema}[section]
\newtheorem{definicao}[teorema]{Definição}
\newtheorem{proposicao}{Proposição}[section]

%preâmbulo
\title{Esse é o título do 1º artigo da aula 3}
\author{César Antônio de Magalhães}
\date{\today}

%conteúdo
\begin{document}
	
	\section{Primeira parte}
	\begin{teorema}
		Todo quadrado tem quatro ângulos retos.
	\end{teorema}
	
	\begin{definicao}
		O quadrado é um polígono de quatro lados.
	\end{definicao}
		
	\begin{teorema}
		Todo triângulo tem três lados.
	\end{teorema}
	
	\begin{proposicao}[Pentágono]
		Um pentágono tem cinco lados 
	\end{proposicao}
	
	\section{Segunda parte}
	
	\begin{teorema}
		Todo quadrado tem quatro ângulos retos.
	\end{teorema}
	
	\begin{definicao}
		O quadrado é um polígono de quatro lados.
	\end{definicao}
	
	\begin{teorema}
		Todo triângulo tem três lados.
	\end{teorema}
	
	\begin{proposicao}
		Um pentágono tem cinco lados 
	\end{proposicao}

\end{document}

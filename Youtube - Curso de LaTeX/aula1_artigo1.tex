\documentclass[12pt, a4paper, twocolumn]{article}
\usepackage[brazilian]{babel} %Traduz o documento para o português brasileiro.
\usepackage[utf8]{inputenc} %Reconhece à acentuação
\usepackage{lipsum} %Gera texto aleatório
\usepackage{ragged2e} %Mais opções de alinhamento

%preâmbulo
\title{Esse é o título do meu artigo}
\author{César Antônio de Magalhães \thanks{Agradeço a minha mãe}}
\date{\today}

%conteúdo
\begin{document}
	
\begin{titlepage}
	\maketitle
\end{titlepage}

\begin{abstract}
	\lipsum[1]
\end{abstract}

\section{Primeiro tema}

Este é o texto do meu documento.
\\ Parágrafo sem indentação.
\newline Outra parágrafo sem indentação. 
\par Aqui se inicia outro parágrafo. 

 Mais texto que se inicia em novo parágrafo. 

\section{Texto alinhado ao centro}

\begin{center}
	\lipsum[1]
\end{center}

\section{Texto alinhado à esquerda}

\begin{flushleft}
	\lipsum[1]
\end{flushleft}

\section{Texto alinhado à direita}

\begin{flushright}
	\lipsum[1]
\end{flushright}

\section{Outro texto alinhado à esquerda}
\raggedright
\lipsum[1]

\section{Outro texto alinhado à direita}
\raggedleft
\lipsum[1]

\justifying

\section{Texto aleatório}

\lipsum

\end{document}

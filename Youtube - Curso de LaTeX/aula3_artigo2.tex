\documentclass[12pt, a4paper, openany]{book}
\usepackage[brazilian]{babel} %Traduz o documento para o português brasileiro.
\usepackage[utf8]{inputenc} %Reconhece à acentuação
\usepackage{lipsum} %Gera texto aleatório

%preâmbulo
\title{Esse é o título do 1º artigo da aula 3}
\author{César Antônio de Magalhães}
\date{\today}

%conteúdo
\begin{document}
	\renewcommand{\contentsname}{Índice}
	%\addcontentsline{toc}{chapter}[Introdução]	
	\tableofcontents
	
	\part{Primeira grande parte}	
		%\chapter*{Introdução}
	\lipsum

\chapter{Primeiro capitulo}		
	\section{Primeiro teste}
		\lipsum[1]

\section{Segundo teste}	
	\subsection{Comentário}
		\lipsum[2]

		\subsubsection{Mais comentários}
			\lipsum[3]

			\paragraph{Primeiro parágrafo}
				\lipsum[4]

				\subparagraph{Primeiro subparágrafo}
					\lipsum[5]
		\chapter*{Introdução}
	\lipsum

\chapter{Primeiro capitulo}		
	\section{Primeiro teste}
		\lipsum[1]

\section{Segundo teste}	
	\subsection{Comentário}
		\lipsum[2]

		\subsubsection{Mais comentários}
			\lipsum[3]

			\paragraph{Primeiro parágrafo}
				\lipsum[4]

				\subparagraph{Primeiro subparágrafo}
					\lipsum[5]	
			
	\part{Segunda grande parte}	
		%\chapter{Mais trabalho}
\lipsum
	
\appendix
\chapter{Conclusão}
	\lipsum
		\chapter{Mais trabalho}
\lipsum
	
\appendix
\chapter{Conclusão}
	\lipsum
							
\end{document}

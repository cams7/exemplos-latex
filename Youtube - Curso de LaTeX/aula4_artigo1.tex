\documentclass[12pt, a4paper]{article}
\usepackage[brazilian]{babel} %Traduz o documento para o português brasileiro.
\usepackage[utf8]{inputenc} %Reconhece à acentuação
\usepackage{lipsum} %Gera texto aleatório

%preâmbulo
\title{Esse é o título do 1º artigo da aula 3}
\author{César Antônio de Magalhães}
\date{\today}

%conteúdo
\begin{document}
		
	\begin{tabular}{|l|@{}l@{}|p{4cm}|c|r|l@{Teste}|}
		\hline
		\multicolumn{2}{|c|}{Dados 1} & \multicolumn{2}{c|}{dados 2} &  & \\ \hline
		Status  & Jogador             & Posição     & Gols marcados  & Faltas & \\ \hline
		Reserva & João                & Atacante    &       5        &      2 & \\ \cline{2-6}
		        & José                & Lateral     &       2        &      1 & \\ \cline{2-6}
		        & Mário               & Centravante &       4        &      1 & \\ \hline
	\end{tabular}
	
	Outra tabela
	
	\begin{tabular}{|c|c|}
		\hline 
		Alimentos & Nota \\ 
		\hline 
		Laranja & 10 \\ 
		\hline 
		Abacate  & 9 \\ 
		\hline 
	\end{tabular} 
	
	\tableofcontents
	
	\listoftables
	
	\section[Lista de exercícios]{Exercícios que demoram muito para serem feitos, e que provavelmente não saberei fazer}
	
	\begin{table}[htbp]
		\centering	
		\begin{tabular}{lr@{.}l}
			Expressão       & \multicolumn{2}{c}{Valor} \\ \hline
			$\pi$           &     3 & 1416              \\
			$\pi^\pi$       &    36 & 46                \\
			$(\pi^\pi)^\pi$ & 80662 & 7                 \\ \hline
		\end{tabular}
		
		\caption{Tabela de exercícios I}
	\end{table}
	
	\lipsum[1]
	
	\begin{table}[htbp]
		\centering	
		\begin{tabular}{lr@{.}l}
			Expressão       & \multicolumn{2}{c}{Valor} \\ \hline
			$\pi$           &     3 & 1416              \\
			$\pi^\pi$       &    36 & 46                \\
			$(\pi^\pi)^\pi$ & 80662 & 7                 \\ \hline
		\end{tabular}
		
		\caption{Tabela de exercícios II}
	\end{table}
	
	\lipsum[2]
	
	\begin{table}[htbp]
		\centering	
		\begin{tabular}{lr@{.}l}
			Expressão       & \multicolumn{2}{c}{Valor} \\ \hline
			$\pi$           &     3 & 1416              \\
			$\pi^\pi$       &    36 & 46                \\
			$(\pi^\pi)^\pi$ & 80662 & 7                 \\ \hline
		\end{tabular}
		
		\caption{Tabela de exercícios III}
	\end{table}
	
	\lipsum[3]
	
	\begin{table}[htbp]
		\centering	
		\begin{tabular}{lr@{.}l}
			Expressão       & \multicolumn{2}{c}{Valor} \\ \hline
			$\pi$           &     3 & 1416              \\
			$\pi^\pi$       &    36 & 46                \\
			$(\pi^\pi)^\pi$ & 80662 & 7                 \\ \hline
		\end{tabular}
		
		\caption{Tabela de exercícios IV}
	\end{table}
	
	\lipsum
	
	\begin{table}[htbp]
		\centering	
		\begin{tabular}{lr@{.}l}
			Expressão       & \multicolumn{2}{c}{Valor} \\ \hline
			$\pi$           &     3 & 1416              \\
			$\pi^\pi$       &    36 & 46                \\
			$(\pi^\pi)^\pi$ & 80662 & 7                 \\ \hline
		\end{tabular}
		
		\caption{Tabela de exercícios V}
	\end{table}
	
	\lipsum
	
	\begin{table}[htbp]
		\centering	
		\begin{tabular}{lr@{.}l}
			Expressão       & \multicolumn{2}{c}{Valor} \\ \hline
			$\pi$           &     3 & 1416              \\
			$\pi^\pi$       &    36 & 46                \\
			$(\pi^\pi)^\pi$ & 80662 & 7                 \\ \hline
		\end{tabular}
		
		\caption[Tabela de exercícios VI]{Tabela de exercícios que demoram muito para serem feitos, e que provavelmente não saberei fazer}
	\end{table}

\end{document}

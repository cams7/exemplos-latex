%Preâmbulo
\documentclass[12pt,a4paper]{article}
\usepackage[brazilian]{babel}
\usepackage[utf8]{inputenc}
\usepackage[left=2cm,right=2cm,top=2cm,bottom=2cm]{geometry}
\usepackage{amsmath,amsfonts,amssymb}
\DeclareMathOperator{\sen}{sen}
\DeclareMathOperator{\tg}{tg}
\DeclareMathOperator{\cossec}{cossec}

\title{Notações de Geometria Analítica}
\author{César Antônio de Magalhães}
\date{\today}

%Corpo do texto
\begin{document}
	\begin{enumerate}
		\item Seja o segmento $\overline{AB}$. A partir dele podemos definir os segmentos orientados
		$\overrightarrow{AB}$ e $\overrightarrow{BA}$.
		\item Seja $\vec{u}$.
		\item Sejam os vetores $\vec{u} = (1;\, -1;\, 2)$ e $\vec{2} = (2;\, 5;\, -4)$. Calcule o seguinte.
		\begin{enumerate}
			\item Produto interno: $\vec{u} \cdot \vec{v}$ ou $\langle \vec{u},\, \vec{v} \rangle$
			\item Produto vetorial: $\vec{u} \times \vec{v}$
			\item Módulo de vetor: $|\vec{u}|$ , $\|\vec{u}\|$ ou $\left\| \overrightarrow{AB} \right\|$
		\end{enumerate}
		\item Sejam os vetores $\vec{u} = (1;\, -1;\, 2)$ e $\vec{2} = (2;\, 5;\, -4)$. Verifique se $\vec{u} \perp \vec{u}$.
		\item Sejam os planos $\alpha : x - 2y + 6z - 3 = 0$ e $\beta : x - 2y -3 = 0$
		\item Sejam os vetores $\vec{u} = (x_0;\, y_0;\, z_0)$ e $\vec{v} = (x_1;\, y_1;\, z_1)$. Temos que: 
		$$\vec{u} \times \vec{v} = 
		\begin{vmatrix}
			\vec{i} & \vec{j} & \vec{k} \\
			x_0 & y_0 & z_0 \\
			x_1 & y_1 & z_1
		\end{vmatrix}$$		
	\end{enumerate}	
\end{document}

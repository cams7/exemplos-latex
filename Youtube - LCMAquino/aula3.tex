%Preâmbulo
\documentclass[12pt,a4paper]{article}
\usepackage[brazilian]{babel}
\usepackage[utf8]{inputenc}
\usepackage{amsmath,amsfonts,amssymb}
\usepackage[left=2cm,right=2cm,top=2cm,bottom=2cm]{geometry}

\title{Notações de conjunto}
\author{César Antônio de Magalhães}
\date{\today}

%Corpo do texto
\begin{document}
	\begin{enumerate}
		\item Sejam os conjuntos $A = \{1;\, 2;\, 3;\, 4\}$,
		$B = \{x \in \mathbb{Z} \,|\, -2 \leq x < 4\}$ e
		$C = \{x \in \mathbb{N} \,|\, x \geq 2\}$. Responda aos itens abaixo.
		\begin{enumerate}
			\item $A \cap B$
			\item $B \cup C$
			\item $A - C$
			\item $C \setminus B$
		\end{enumerate}
		
		\item Classifique em verdadeiro ou falso.
		\begin{enumerate}
			\item $\mathbb{Z} \not\subset \mathbb{N}$
			\item $\mathbb{R} \not\supset \mathbb{Q}$
			\item $0 \not\in \mathbb{R} \setminus\mathbb{Q}$
			\item $\forall x \in \mathbb{N}$, temos $x \geq 0$
			\item $\exists x \in \mathbb{R}$, tal que $\sqrt{x} \not\in \mathbb{R}$
			\item $7 \not\in \{x \in \mathbb{N} \,|\, x\textrm{ é par}\}$
			\item $-5 \in \mathbb{R}^*_+$
			\item $0 \in \varnothing$
		\end{enumerate}
	\end{enumerate}		
\end{document}

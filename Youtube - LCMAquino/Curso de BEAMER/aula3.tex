\documentclass[aspectratio=169]{beamer}
\usetheme{Berlin}
\usepackage[brazilian]{babel}
\usepackage[utf8]{inputenc}
\usepackage{graphicx}
\graphicspath{{img/}}
\usepackage{amsmath,amsfonts,amssymb}
\setbeamercovered{transparent}

\title{Estrutura da Apresentação}
\subtitle{Curso de BEAMER}
\author[César]{César A. de Magalhães}
\institute[UNOPAR]{Universidade do Norte do Paraná \\ https://vestibular.unoparead.com.br}
\date{\today}
\logo{\includegraphics[scale=0.25]{ctan-leao.jpg}}

\AtBeginPart{	
	\begin{frame}{Sumário}
		\tableofcontents[pausesections,pausesubsections]
		%\tableofcontents[currentpart]
	\end{frame}
}

\AtBeginSection{
	\begin{frame}
		\tableofcontents[currentsection, hideallsubsections]
	\end{frame}
}
%\AtBeginSubsection{
%	\begin{frame}
%		\tableofcontents[currentsubsection, hideothersubsections]
%	\end{frame}
%}

\begin{document}
	\begin{frame}
		\titlepage
	\end{frame}
			
	\part{Primeira apresentação}
		\section[Introdução]{Introdução à primeira apresentação}
			\begin{frame}
				\frametitle{Aqui é o título do quadro 1}
				\framesubtitle{Aqui é o subtítulo do quadro 1}
				Esse e o quadro 1.
			\end{frame}	
		
		\section[Metodologia]{Metodologia da primeira apresentação}
			\subsection{Uma subseção da metodologia}
				\begin{frame}
					\frametitle{Aqui é o título do quadro 2}
					\framesubtitle{Aqui é o subtítulo do quadro 2}
					Esse e o quadro 2 \cite{meuartigo}.
				\end{frame}	
			
			\subsection{Uma outra subseção da metodologia}
				\begin{frame}
					\frametitle{Aqui é o título do quadro 3}
					\framesubtitle{Aqui é o subtítulo do quadro 3}
					Esse e o quadro 3.
				\end{frame}	
		
		\section[Discussão]{Discussão da primeira apresentação}
			\subsection{Uma subseção da discussão}
				\begin{frame}
					\frametitle{Aqui é o título do quadro 4}
					\framesubtitle{Aqui é o subtítulo do quadro 4}
					Esse e o quadro 4 \cite{meulivro}.
				\end{frame}
				\begin{frame}
					\frametitle{Aqui é o título do quadro 5}
					\framesubtitle{Aqui é o subtítulo do quadro 5}
					Esse e o quadro 5.
				\end{frame}
			
				\subsection{Uma outra subseção da discussão}
					\begin{frame}
						\frametitle{Aqui é o título do quadro 6}
						\framesubtitle{Aqui é o subtítulo do quadro 6}
						Esse e o quadro 6.
					\end{frame}
			
			\subsubsection{Subseção da subseção discussão}
				\begin{frame}
					\frametitle{Aqui é o título do quadro 7}
					\framesubtitle{Aqui é o subtítulo do quadro 7}
					Esse e o quadro 7.
				\end{frame}
		
		\section[Conclusão]{Conclusão da primeira apresentação}
			\begin{frame}
				\frametitle{Aqui é o título do quadro 8}
				\framesubtitle{Aqui é o subtítulo do quadro 8}
				Esse e o quadro 8.
			\end{frame}
			\begin{frame}
				\frametitle{Aqui é o título do quadro 9}
				\framesubtitle{Aqui é o subtítulo do quadro 9}
				Esse e o quadro 9.
			\end{frame}
		
	\part{Segunda apresentação}
		\section[Introdução]{Introdução à segunda apresentação}
			\begin{frame}
				\frametitle{Aqui é o título do quadro 1}
				\framesubtitle{Aqui é o subtítulo do quadro 1}
				Esse e o quadro 1.
			\end{frame}	
		
		\section[Metodologia]{Metodologia da segunda apresentação}
			\subsection{Uma subseção da metodologia}
				\begin{frame}
					\frametitle{Aqui é o título do quadro 2}
					\framesubtitle{Aqui é o subtítulo do quadro 2}
					Esse e o quadro 2 \cite{meuartigo}.
				\end{frame}	
			
			\subsection{Uma outra subseção da metodologia}
				\begin{frame}
					\frametitle{Aqui é o título do quadro 3}
					\framesubtitle{Aqui é o subtítulo do quadro 3}
					Esse e o quadro 3.
				\end{frame}	
		
		\section[Discussão]{Discussão da segunda apresentação}
			\subsection{Uma subseção da discussão}
				\begin{frame}
					\frametitle{Aqui é o título do quadro 4}
					\framesubtitle{Aqui é o subtítulo do quadro 4}
					Esse e o quadro 4 \cite{meulivro}.
				\end{frame}
				\begin{frame}
					\frametitle{Aqui é o título do quadro 5}
					\framesubtitle{Aqui é o subtítulo do quadro 5}
					Esse e o quadro 5.
				\end{frame}
			
			\subsection{Uma outra subseção da discussão}
				\begin{frame}
					\frametitle{Aqui é o título do quadro 6}
					\framesubtitle{Aqui é o subtítulo do quadro 6}
					Esse e o quadro 6.
				\end{frame}
				
				\subsubsection{Subseção da subseção discussão}
					\begin{frame}
						\frametitle{Aqui é o título do quadro 7}
						\framesubtitle{Aqui é o subtítulo do quadro 7}
						Esse e o quadro 7.
					\end{frame}
		
		\section[Conclusão]{Conclusão da segunda apresentação}
			\begin{frame}
				\frametitle{Aqui é o título do quadro 8}
				\framesubtitle{Aqui é o subtítulo do quadro 8}
				Esse e o quadro 8.
			\end{frame}
			\begin{frame}
				\frametitle{Aqui é o título do quadro 9}
				\framesubtitle{Aqui é o subtítulo do quadro 9}
				Esse e o quadro 9.
			\end{frame}
	
	\section*{Referências}
	\begin{frame}
		\bibliographystyle{apalike}
		\bibliography{referencia}
	\end{frame}
\end{document}
\documentclass[aspectratio=169]{beamer}
\usetheme{PaloAlto}
\usepackage[brazilian]{babel}
\usepackage[utf8]{inputenc}
\usepackage{graphicx}
\graphicspath{{img/}}
\usepackage{amsmath,amsfonts,amssymb}
\setbeamercovered{transparent}

\title{Curso de BEAMER}
\subtitle{Subtítulo da apresentação}
\author[César \and Jorge \and Clarrise \and Luis]{César A. de Magalhães \and Jorge Amado \and Clarrise Lispector \and Luis Fernado Verissimo}
\institute[UNOPAR]{Universidade do Norte do Paraná \\ https://vestibular.unoparead.com.br}
\date{\today}
\logo{\includegraphics[scale=0.25]{ctan-leao.jpg}}

\begin{document}
	\begin{frame}
		\titlepage
	\end{frame}
	\begin{frame}
		\frametitle{Equação do segundo grau}
		\framesubtitle{Primeiro quadro - pause}
		
		Vamos resolver uma equação do segundo grau:
		$$ax^2 + bx + c = 0$$\pause
		
		Primeiro identifique os coeficientes $a$, $b$ e $c$.\pause
		
		Em seguida, calcule o valor de: 
		$$\Delta = b^2 - 4ac$$\pause
		
		Calcule a primeira raiz:
		$$x_1 = \dfrac{-b + \sqrt{\Delta}}{2a}$$\pause
		
		Calcule a segunda raiz:
		$$x_2 = \dfrac{-b - \sqrt{\Delta}}{2a}$$\pause
	\end{frame}
	\begin{frame}
		\frametitle{Equação do segundo grau}
		\framesubtitle{Segundo quadro - uncover}
		
		\uncover<1>{Vamos resolver uma equação do segundo grau:
			$$ax^2 + bx + c = 0$$}
		
		\uncover<2>{Primeiro identifique os coeficientes $a$, $b$ e $c$.}
		
		\uncover<3>{Em seguida, calcule o valor de: 
			$$\Delta = b^2 - 4ac$$}
		
		\uncover<4>{Calcule a primeira raiz:
			$$x_1 = \dfrac{-b + \sqrt{\Delta}}{2a}$$}
		
		\uncover<5>{Calcule a segunda raiz:
			$$x_2 = \dfrac{-b - \sqrt{\Delta}}{2a}$$}
	\end{frame}
	\begin{frame}
		\frametitle{Equação do segundo grau}
		\framesubtitle{Terceiro quadro - visible}
		
		\visible<1>{Vamos resolver uma equação do segundo grau:
			$$ax^2 + bx + c = 0$$}
		
		\visible<2>{Primeiro identifique os coeficientes $a$, $b$ e $c$.}
		
		\visible<3>{Em seguida, calcule o valor de: 
			$$\Delta = b^2 - 4ac$$}
		
		\visible<4>{Calcule a primeira raiz:
			$$x_1 = \dfrac{-b + \sqrt{\Delta}}{2a}$$}
		
		\visible<5>{Calcule a segunda raiz:
			$$x_2 = \dfrac{-b - \sqrt{\Delta}}{2a}$$}
	\end{frame}
	\begin{frame}
		\frametitle{Equação do segundo grau}
		\framesubtitle{Quarto quadro - only}
		
		\only<1>{Vamos resolver uma equação do segundo grau:
			$$ax^2 + bx + c = 0$$}
		
		\only<2>{Primeiro identifique os coeficientes $a$, $b$ e $c$.}
		
		\only<3>{Em seguida, calcule o valor de: 
			$$\Delta = b^2 - 4ac$$}
		
		\only<4>{Calcule a primeira raiz:
			$$x_1 = \dfrac{-b + \sqrt{\Delta}}{2a}$$}
		
		\only<5>{Calcule a segunda raiz:
			$$x_2 = \dfrac{-b - \sqrt{\Delta}}{2a}$$}
	\end{frame}
	\begin{frame}
		\frametitle{Equação do segundo grau}
		\framesubtitle{Quinto quadro - itemize}
		
		\begin{itemize}
			\item<1- | alert@1> Vamos resolver uma equação do segundo grau:
			$$ax^2 + bx + c = 0$$			
			\item<2- | alert@2> Primeiro identifique os coeficientes $a$, $b$ e $c$.			
			\item<3- | alert@3> Em seguida, calcule o valor de: 
			$$\Delta = b^2 - 4ac$$			
			\item<4- | alert@4> Calcule a primeira raiz:
			$$x_1 = \dfrac{-b + \sqrt{\Delta}}{2a}$$			
			\item<5- | alert@5> Calcule a segunda raiz:
			$$x_2 = \dfrac{-b - \sqrt{\Delta}}{2a}$$
		\end{itemize}		
	\end{frame}	
	\begin{frame}
		\frametitle{Equação do segundo grau}
		\framesubtitle{Sexto quadro - itemize}
		
		\begin{itemize}[<+- | alert@+>]
			\item Vamos resolver uma equação do segundo grau:
			$$ax^2 + bx + c = 0$$			
			\item Primeiro identifique os coeficientes $a$, $b$ e $c$.			
			\item Em seguida, calcule o valor de: 
			$$\Delta = b^2 - 4ac$$			
			\item Calcule a primeira raiz:
			$$x_1 = \dfrac{-b + \sqrt{\Delta}}{2a}$$			
			\item Calcule a segunda raiz:
			$$x_2 = \dfrac{-b - \sqrt{\Delta}}{2a}$$
		\end{itemize}		
	\end{frame}
	\begin{frame}
		\frametitle{Equação do segundo grau}
		\framesubtitle{Sétimo quadro - block}
		
		\begin{block}<1 - | only@1>{}		
			Vamos resolver uma equação do segundo grau:
			$$ax^2 + bx + c = 0$$
		\end{block}
		
		\begin{block}<2 - | only@2>{\ }
			Primeiro identifique os coeficientes $a$, $b$ e $c$.
		\end{block}
		
		\begin{block}<3 - | only@3>{\ }
			Em seguida, calcule o valor de: 
			$$\Delta = b^2 - 4ac$$
		\end{block}
		
		\begin{block}<4 - | only@4>{Primeira raiz}
			Calcule a primeira raiz:
			$$x_1 = \dfrac{-b + \sqrt{\Delta}}{2a}$$
		\end{block}
		
		\begin{block}<5 - | only@5>{Segunda raiz}
			Calcule a segunda raiz:
			$$x_2 = \dfrac{-b - \sqrt{\Delta}}{2a}$$
		\end{block}
	\end{frame}
	\begin{frame}{Oitavo quadro - figure}
		Aqui vem o texto
		
		\begin{figure}
			\centering
			\includegraphics<1>[scale=0.5]{cabra.png}
			\includegraphics<2>[scale=0.5]{cachorro.png}
			\includegraphics<3>[scale=0.2]{elefante.png}
			\includegraphics<4>[scale=0.5]{hipopotamo.png}
			\includegraphics<5>[scale=0.5]{leao.png}
		\end{figure}
	\end{frame}
\end{document}